\startAbstractCz
%shrnutí celé práce
%popisný abstrakt vysvětluje účel, cíl a metody 100-200 slov
%proč jste se rozhodli psát na toto téma? proč je tenhle "výzkum" důležitý?
%metody
%výsledky?

%    Čím se v textu práce zabýváte (jakým problémem)?
%    Jaké metody jste použili na zjištění / výzkum / řešení problému?
%    Jakých výsledků jste dosáhli?


%Tato práce řeší.
%Cílem této práce je.
%Zaměřil jsem se na.

%Zvolený problém jsem vyřešil pomocí toho a toho.
%V řešení bylo použito metody té, postupu toho a analýzy oné.
%Práce představuje algoritmus takový, který.
%Data jsem zpracovával pomocí těch a těchto nástrojů a provedl vyhodnocení takové.
%Podstatou našeho algoritmu je.

%Podařilo se dosáhnout úspěšnosti 87,3%.
%V práci jsme vytvořili systém, který.
%Vytvořené řešení poskytuje ty a ty možnosti.
%Provedeným výzkumem jsme zjistili, že.

%Přínosem této práce je.
%Hlavním zjištěním je.
%Hlavním výsledkem je.
%Na základě zjištěných údajů je možné.
%Výsledky této práce umožňují.

  Text abstraktu česky\dots
\stopAbstractCz

\startAbstractEn
  Text of abstract in English\dots
\stopAbstractEn

\endinput
%%
%% End of file `abstract.tex'.
