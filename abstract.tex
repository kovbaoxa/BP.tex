\startAbstractCz
%shrnutí celé práce
%popisný abstrakt vysvětluje účel, cíl a metody 100-200 slov
%proč jste se rozhodli psát na toto téma? proč je tenhle "výzkum" důležitý?
%metody
%výsledky?

%    Čím se v textu práce zabýváte (jakým problémem)?
%    Jaké metody jste použili na zjištění / výzkum / řešení problému?
%    Jakých výsledků jste dosáhli?


Tato práce řeší detekci ruky pomocí kamery Kinect a rozpoznávání gest. Cílem této práce je vytvořit podklad pro řízení robotu, který bude interagovat s člověkem. Zaměřila jsem se na charakteristické rysy ruky a rozpoložení prstů.
Práce představuje algoritmus, který využívá pouze euklidovské vzdálenosti. Data z kamery jsem zpracovávala pomocí openFrameworks s využitím doplňku ofxKinectV2.

%Podstatou našeho algoritmu je.

%Podařilo se dosáhnout úspěšnosti 87,3%.
%V práci jsme vytvořili systém, který.
%Vytvořené řešení poskytuje ty a ty možnosti.
%Provedeným výzkumem jsme zjistili, že.

%Přínosem této práce je.
%Hlavním zjištěním je.
%Hlavním výsledkem je.
%Na základě zjištěných údajů je možné.
%Výsledky této práce umožňují.
\stopAbstractCz

\startAbstractEn
This thesis deals with hand detection and gesture recognition from video stream captured by Kinect camera. The aim of thesis is to create data to control robot for human-robot interaction. Focus was on features of the hand and fingers. Thesis presents an algorithm based on euclidean distances. Data was processed using openFrameworks software with ofxKinectV2 addon.
\stopAbstractEn

\endinput
%%
%% End of file `abstract.tex'.
