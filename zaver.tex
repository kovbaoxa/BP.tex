\chapter{Závěr}
Vybraný postup implementace zaručil určité výhody oproti jiným algoritmům. Mezi hlavní výhody paří, že nejsou potřeba žádné jiné věci, než je kamera. Člověk ovládající robota nemusí  být oproti žádnému konkrétnímu pozadí, nemusí mít na rukou rukavice nebo mít jiné specifické náležitosti. Úvodní kalibrace také není nutná. 

%možnosti budoucího využití
Jelikož je rozpoznávání vykonáváno v binárním obraze, je možná kombinace více programů nebo metod relativně jednoduchá. Zároveň lze program použít i s jinými hloubkovými kamerami. Pokud se algoritmus zkombinuje s využíváním obrazu z barevné kamery, dalo lze detekci provádět spolehlivěji.
Do programu lze také přidat nová gesta, odstranit stávající nebo změnit celkový pohled, na základě kterého se definují gesta. To lze udělat z pohledu přesnějšího rozpoznávání, tak i z pohledu, jak se jednotlivá gesta uvažují (počet prstů versus konkrétní prsty). 
%zhodnocení výsledků
%odůvodnění proč a jak

TODO zhodnocení výsledků - statistiky


\endinput
%%
%% End of file `ch01.tex'.
