\chapter{Úvod}
%-full-textové vyhldávání (key-words?)
%seznámení s problematikou - současná situace, motivace
V dnešní době robotizace, kdy lidé přicházejí na výhody využívání inteligentních zařízení, je snaha rozšířit jejich využití co nejvíce. Ovládání gesty značně pomůže intuitivnímu řízení robotů a usnadní jejich využití.
Z pohledu spolupráce běžného uživatele s robotem se řadí řízení zařízení za pomoci kamery mezi nejjednodušší způsoby, jelikož kamery jsou relativně levné a dostupné. Není k tomu potřeba nic jiného, než co většina populace (alespoň cílové skupiny - lidí používajíví počítače) už má. Jedná se o jednodušší variantu i s ohledem na implementační čas.

Mnoho podobných aplikací již existuje, nebo se zaměřuje na jednotlivé podčásti, často ale vyžadují složitou instalaci a konfiguraci programů. Velké množstí pak ani nepodporuje zpracování s malou časovou odezvou a jsou výpočetně složité a časové náročné.
%cíle BP
Jelikož má být výsledný algoritmus použitelný pro robota, který  bude  následovat člověka, musí splňovat několik požadavků. Jedná se o systém reálného času a tam musí být odezva dostatečně malá s ohledem na člověka. Zvolená gesta musí být intuitivní a snadno proveditelná. 		%další požadavky 

%odůvodnění proč a jak
Následně zvolené přístupy v této práci jsou psány tak, aby měly co nejmenší odezvu. 
V aktuálních aplikacích se obvykle využívají strojově naučené algoritmy. Pro snazší dosáhnutí rychlejšího zpracování obrazu, které vyhovuje požadavkům, lze využít vlastností obrazu, jako jsou vzdálenosti a úhly k rozpoznávání ruky.
%popsání logického členění BP



\endinput