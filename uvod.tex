\chapter{Úvod}
V dnešní době robotizace, kdy lidé přicházejí na výhody využívání inteligentních \\(VLOŽIT REFERENCI podle jaké definice)\\ zařízení, je snaha rozšířit jejich využití co nejvíce. \\POZN:vložit delší povídání o tom co se již děje atd?\\
Řízení zařízen za pomoci kamery se řádí mezi nejjednodušší způsoby, jelikož kamery jsou relativně levné a dostupné. Není k tomu potřeba nic jiného, než co většina populace (alespoň cílové skupiny - lidí používajíví počítače) už má. Jedná se o jednodušší variantu i s ohledem na implementační čas.


\section{Důvody}
Mnoho podobných aplikací již existuje, nebo se zaměřuje na jednotlivé podčásti, často ale vyžadují složitou instalaci a konfiguraci programů. Velké množstí pak ani nepodporuje zpracování v reálném čase a jsou výpočetně složité a časové náročné. Cílem bakalářské práce je vytvořit program relativně jednoduše nastavitelná s tak rychlou odezvou, jak to jen bude robustnost dovolovat.

Následně řešené přistupy jsou voleny tak, aby měly co nejmenší odezvu. S lepší robustností již lze navázat na konci. %FORMULACE?????

\endinput