\chapter{Úvod}
%-full-textové vyhldávání (key-words?)
%seznámení s problematikou - současná situace, motivace
V dnešní době robotizace, kdy lidé přicházejí na výhody využívání inteligentních zařízení, je snaha rozšířit jejich využití co nejvíce. Ovládání gesty značně pomůže intuitivnímu řízení robotů a usnadní jejich využití.
Řízení zařízení za pomoci kamery se řadí mezi nejjednodušší způsoby, jelikož kamery jsou relativně levné a dostupné. Není k tomu potřeba nic jiného, než co většina populace (alespoň cílové skupiny - lidí používajíví počítače) už má. Jedná se o jednodušší variantu i s ohledem na implementační čas.

Mnoho podobných aplikací již existuje, nebo se zaměřuje na jednotlivé podčásti, často ale vyžadují složitou instalaci a konfiguraci programů. Velké množstí pak ani nepodporuje zpracování v reálném čase a jsou výpočetně složité a časové náročné.
%cíle BP
Jelikož má být výsledný algoritmus použitelný pro robota, který  bude  následovat člověka, musí splňovat několik požadavků. Časová odezva musí být real-time v časech vyhovujících člověku. Zvolená gesta musí být intuitivní a snadno proveditelná. 		%další požadavky 

%odůvodnění proč a jak
Následně řešené přistupy jsou voleny tak, aby měly co nejmenší odezvu. 
V aktuálních aplikacích se nejčastěji využívají strojově naučené algoritmy. Pokud se ale k rozpoznávání využijí vlastnosti obrazu, jako jsou vzdálenosti a úhly, lze snáze dosáhnout časů zpracování, která vyhovují požadavkům.
%popsání logického členění BP






\endinput